\documentclass{article}
\usepackage{float}
\usepackage{amsmath}
\usepackage{graphicx}

\title{2DT904 Assignment 2}
\author{Samuel Berg}
\date{December 2024}

\begin{document}

\maketitle

\tableofcontents


\newpage
\section{Rasterization}

My point is $p = (x_{\text{viewport}}, y_{\text{viewport}}) = (2,5)$ \\
Therefore, \\
\hspace*{10pt} my pixel center is $P = (x_p, y_p) = (x_{\text{viewport}} + 0.5, y_{\text{viewport}} + 0.5)= (2.5, 5.5)$ 

$$
x_{\text{clip}} = \frac{2 \cdot x_p}{\text{width}} - 1, \quad y_{\text{clip}} = 1 - \frac{2 \cdot y_p}{\text{height}}
$$

$$
x_{\text{clip}} = \frac{2 \cdot 2.5}{8} - 1= -0.375, \quad y_{\text{clip}} = 1 - \frac{2 \cdot 5.5}{8} = -0.375
$$
\\
For vertices $V_1 = (-0.8, 1), \quad V_2 = (0.9, 0.6), \quad V_3 = (-0.2, -1)$:
$$
E(x, y) = (y_2 - y_1)(x_p - x_1) - (x_2 - x_1)(y_p - y_1)
$$
$$
E_{12} = (0.6 - 1)(-0.375 - (-0.8)) - (0.9 - (-0.8))(-0.375 - 1) = 2.1675
$$
$$
E_{23} = (-1 - 0.6)(-0.5 - 0.9) - (-0.2 - 0.9)(-0.25 - 0.6) = 0.9675
$$
$$
E_{31} = (1 - (-1))(-0.5 - (-0.2)) - (-0.8 - (-0.2))(-0.25 - (-1)) = 0.025
$$

$$
E_{12} = 2.1675 \quad (\text{positive}), \quad E_{23} = 0.9675 \quad (\text{positive}), \quad E_{31} = 0.025 \quad (\text{positive})
$$

Since all $E$ are positive, the pixel $(-0.375, -0.375)$ lies inside the triangle.


\newpage
\section{Interpolation}

$$
P = (x_p, y_p) = (-0.375, -0.375)
$$

$$
A_{\text{total}} = \frac{1}{2} \left| x_1(y_2 - y_3) + x_2(y_3 - y_1) + x_3(y_1 - y_2) \right|
$$

For vertices $V_1 = (-0.8, 1), \quad V_2 = (0.9, 0.6), \quad V_3 = (-0.2, -1)$:

$$
A_{\text{total}} = \frac{1}{2} \left| (-0.8)(0.6 - (-1)) + 0.9((-1) - 1) + (-0.2)(1 - 0.6) \right| = 1.58
$$

$$
A_{P, V_2, V_3} = \frac{1}{2} |
(-0.375)(0.6 - (-1)) + 0.9((-1) - (-0.375)) + (-0.2)((-0.375) - 0.6)| = 0.4838
$$
Thus:
$$
\alpha = \frac{A_{P, V_2, V_3}}{A_{\text{total}}} = \frac{0.4838}{1.58} \approx 0.31
$$
\\
$$
A_{V_1, P, V_3} = \frac{1}{2} \left| (-0.8)((-0.375) - (-1)) + (-0.375)((-1) - 1) + (-0.2)(1 - (-0.375)) \right| = 0.0125
$$
Thus:
$$
\beta = \frac{A_{V_1, P, V_3}}{A_{\text{total}}} = \frac{0.0125}{1.58} \approx 0.01
$$
\\
$$
A_{V_1, V_2, P} = \frac{1}{2} \left| (-0.8)(0.6 - (-0.375)) + 0.9((-0.375) - 1) + (-0.375)(1 - 0.6) \right| = 1.0837
$$
Thus:
$$
\gamma = \frac{A_{V_1, V_2, P}}{A_{\text{total}}} = \frac{1.0837}{1.58} \approx 0.69
$$
\\

For $C_1 = (1, 1, 1), \quad C_2 = (0, 0, 0),\quad C_3 = (0.3, 0.3, 1.0)$:
$$
C = \alpha C_1 + \beta C_2 + \gamma C_3
$$
$$
C = 0.31 (1, 1, 1) + 0.01 (0, 0, 0) + 0.69 (0.3, 0.3, 1.0)
$$

Then, each component:
$$
C_x = 0.31(1) + 0.01(0) + 0.69(0.3) \approx 0.52
$$
$$
C_y = 0.31(1) + 0.01(0) + 0.69(0.3) \approx 0.52
$$
$$
C_z = 0.31(1) + 0.01(0) + 0.69(1.0) \approx 1.00
$$

Thus, the interpolated color is:
$$
C = (0.52, 0.52, 1.00)
$$


\newpage
\section{Lighting}

$$
\text{} I = I_{\text{diffuse}} + I_{\text{ambient}}
$$

$$
\text{For light vector} \quad L = (2, 6, 3)
$$
$$L_{\text{norm}} = \frac{L}{\|L\|}$$
$$
\|L\| = \sqrt{2^2 + 6^2 + 3^2} = \sqrt{4 + 36 + 9} = \sqrt{49} = 7
$$

Thus, the normalized light vector is:
$$
L_{\text{norm}} = \left( \frac{2}{7}, \frac{6}{7}, \frac{3}{7} \right)
$$

$$
I_{\text{diffuse}} = k_d \cdot I_d \cdot \max(0, N \cdot L) \cdot C
$$

Where:

$k_{d} = 1.0$ (since the triangle is completely diffuse),

$I_{d} = (1.0, 1.0, 0.7)$ (directional light color),

$N = (0, 0, 1)$ (normal vector).

$C = (0.52, 0.52, 1.00)$ (interpolating light/color)

$$
N \cdot L = (0) \cdot \frac{2}{7} + (0) \cdot \frac{6}{7} + (1) \cdot \frac{3}{7} = \frac{3}{7}
$$

$$
I_{\text{diffuse}} = 1.0 \cdot (1.0, 1.0, 0.7) \cdot \frac{3}{7} \cdot (0.52, 0.52, 1.00) \approx (0.22, 0.22, 0.3)
$$

$$
I_{\text{ambient}} = k_a \cdot I_a \cdot C
$$

Where:

$k_{a} = 1.0$ (full reflection),

$I_{a} = (0.2, 0.2, 0.2)$ (ambient light color).

$C = (0.52, 0.52, 1.00)$ (interpolating light/color)

Thus:
$$
I_{\text{ambient}} = 1.0 \cdot (0.2, 0.2, 0.2) \cdot (0.52, 0.52, 1.00) \approx (0.1, 0.1, 0.2)
$$

Thus:
$$
I = (0.22, 0.22, 0.3) + (0.1, 0.1, 0.2) = (0.33, 0.33, 0.50)
$$


\newpage
\section{Texturing}

\subsection{Sampling}

\subsubsection{a. \textbf{Nearest} filtering}
For $\quad U = 0.42, V = 0.58$
$$
x = U \cdot (M - 1), \quad y = V \cdot (N - 1)
$$
Where $M \cdot N$ the texture's grid size. For the current texture: $M = 16, \quad N = 12$

$$
x = 0.42 \cdot (16 - 1) = 6.3 \approx 6, \quad y = 0.58 \cdot (12 - 1) = 6.38 \approx 6
$$
At $(6, 6)$ in the texture gives $(0.85, 0.65, 0.13)$

\subsubsection{b. \textbf{Bilinear} filtering}
$$
x_0 = floor(U \cdot (M - 1)), \quad y_0 = floor(V \cdot (N - 1)) 
$$
$$
x_1 = min(x_0 + 1, M - 1), \quad y_1 = min(y_0 + 1, N - 1)
$$

$$
x_0 = floor(0.42 \cdot (16 - 1)) = 6, \quad y_0 = floor(0.58 \cdot (12 - 1))  = 6
$$
$$
x_1 = min(6 + 1, 16 - 1) = 7, \quad y_1 = min(6 + 1, 12 - 1) = 7
$$

$$
w_{x_1} = x - x_0, \quad w_{y_1} = y - y_0
$$
$$
w_{x_0} = 1 - w_{x_1}, \quad w_{y_0} = 1 - w_{y_1}
$$

$$
w_{x_1} = 6.3 - 6 = 0.3, \quad w_{y_1} = 6.38 - 6 = 0.38
$$
$$
w_{x_0} = 1 - 0.3 = 0.7, \quad w_{y_0} = 1 - w_{y_1} = 0.62
$$
\\
Affecting Colors:
$$
C_{00} = (x_0, y_0) = (0.85, 0.65, 0.13), \quad C_{01} = (x_0, y_1) = (0.85, 0.65, 0.13)
$$
$$
C_{10} = (x_1, y_0) = (1.00, 0.00, 0.00), \quad C_{11} = (x_1, y_1) = (0.00, 0.00, 1.00)
$$
\\
Bilinear Color:
$$
C = C_{00} \cdot w_{x_0} \cdot w_{y_0} + C_{01} \cdot w_{x_0} \cdot w_{y_1} + C_{10} \cdot w_{x_1} \cdot w_{y_0} + C_{11} \cdot w_{x_1} \cdot w_{y_1}
$$
$$
C = C_{00} \cdot 0.7 \cdot 0.62 + C_{01} \cdot 0.7 \cdot 0.38 + C_{10} \cdot 0.3 \cdot 0.62 + C_{11} \cdot 0.3 \cdot 0.32
$$
$$
C = C_{00} \cdot 0.434 + C_{01} \cdot 0.266 + C_{10} \cdot 0.186 + C_{11} \cdot 0.114 = (0.781, 0.455, 0.205)
$$

$$
C = (0.781, 0.455, 0.205)
$$


\newpage
\subsection{UV-mapping}

To compute UV for Mario's head, we need the bounds of his head according to the original texture. 
\\
$$
\text{Top-left: } p_0 = (1, 0), \text{ Top-right: } p_1 =(11, 0), \text{ Bottom-left: } p_2 = (1, 7),\text{ Bottom-right: } p_3 = (11, 7).
$$
\\
Which gives the following texture:

\begin{figure}[H]
    \centering
    \includegraphics[width=0.5\textwidth]{./img/mario_head.png}
    \caption{Mario's head}
    \label{fig:mario-head}
\end{figure}

$$
U = \frac{\text{pixel}_x}{\text{width} -1}, \quad V = \frac{\text{pixel}_y}{\text{height} - 1}
$$
Where $\text{width} = 12 \text{ and height} = 16$ due to the original texture's size.
\\
\\
For $p_0 = (1, 0)$:
$$
U_0 = \frac{1}{\text{12} -1} \approx 0.0909, \quad V_0 = \frac{0}{16 - 1} = 0
$$
\\
For $p_1 = (11, 0)$:
$$
U_1 = \frac{11}{\text{12} -1} = 1.0, \quad V_1 = \frac{0}{16 - 1} = 0
$$
\\
For $p_2 = (1, 7)$:
$$
U_2 = \frac{1}{\text{12} -1} \approx 0.0909, \quad V_2 = \frac{7}{16 - 1} \approx 0.4667
$$
\\
For $p_3 = (11, 7)$:
$$
U_3 = \frac{11}{\text{12} -1} = 1.0, \quad V_3 = \frac{7}{16 - 1} \approx 0.4667
$$

\newpage
For reference here is the original texture:

\begin{figure}[H]
    \centering
    \includegraphics[width=0.5\textwidth]{./img/mario_texture.png}
    \caption{Mario texture}
    \label{fig:mario}
\end{figure}

\end{document}
