\documentclass[12pt]{article}

\usepackage[nottoc]{tocbibind}
\usepackage[colorlinks=true,
            linkcolor=blue,
            urlcolor=blue,
            citecolor=black]{hyperref}

\begin{document}

\begin{center}
\Large \textbf{Project Plan for Engineering Degree Projects } \\
\large \textbf{Department of Computer Science}
\end{center}

\section*{General Information}
\begin{tabular} {|p{3.8cm}|p{9cm}|} \hline
Title: & Preliminary title of your degree project \\ \hline
External company: & Name of the company (if you do your degree project at an external company) \\ \hline
\end{tabular}

\section*{Persons involved}
\begin{tabular} {|p{2.2cm}|p{4.7cm}|p{5.47cm}|} \hline
Student 1: & Your name here & Your email address here \\ \hline
Student 2: &  &  \\ \hline
\end{tabular}
\\ \vspace*{0.2cm} \\
\begin{tabular} {|p{3.9cm}|p{8.9cm}|} \hline
Supervisor: & Your supervisor at the university (if you already have been appointed a supervisor) \\ \hline
External supervisor: & Your supervisor at the company (if you do your degree project at an external company) \\ \hline
\end{tabular}
\\


Some of the material presented in this project plan should already be available in the project description you handed in when the course started. However, try to come up with more precise and concrete formulations. We expect the entire project plan to be about 4 pages (not including references). \textit{Do not make any changes in this template (apart from removing our instructions) unless it has been approved by your Lnu supervisor. }

\section*{Background}
Describe the background to the problem area of your degree project. Begin with a broad overview of what your 
project is all about, then focus in on the particular problem. You will need references that support your claims 
and the problem. A well written background is guiding the reader from the broad perspective, why there is a need 
for an investigation, and narrowing down to the specific problem. First a subject area and then the area of investigation.

This is also the place where you can give a brief presentation of company/organization you are working with (if any)


** About half a page. **

\section*{Related Work}
Describe what already have been done in this problem domain. Make very brief summaries of 2-3 papers that 
are directly related to your problem domain. \\

\noindent
\textbf{Notice:}  This section should not be extensive. A much larger version of the related work section will be presented in your final report.


\section*{Problem formulation}
Briefly describe the problem you plan to investigate, its limitations, and what results you expect. The problem formulation should be clearly related to your background and you should also point out how it relates to the related works. Try to  specify the knowledge gap, the gap between what is known (related work) and what you plan to do. Although presented as a text, try to be concrete.  About limitations: For example, hardware and software requirements imposed by a company. About expected results: What concrete output do you expect from the project? It can be a  software component fulfilling a list of requirements or a prototype implementation fulfilling certain criteria (e.g. accuracy $>$ 90\%). \\

\noindent
\textbf{Notice:}  Your project might involve handling 2-4 problems. For example: 1) Find the best possible approach to handle a certain problem, 2) Create a prototype implementing the best approach, and 3) Present a series of experiments evaluating the implementation . 


\section*{Motivation}
Motivate why handling the  problems in your problem formulation is important. And not only for a single company/organization,  why is your problem interesting for science, society or industry (in general). \\

\noindent
\textbf{Notice:}  The above sections (Background, related work, problem formulation, motivation) should provide enough information for a reader to understand the more detailed sections that follows. All domain specific concepts used in what follows should have been introduced.

\section*{Milestones and Time Plan}
The problem you shall investigate is broken down into a list of milestones with a rough estimate of when the milestones/tasks are expected to be finished. A milestones is something you shall do within the time frame of your degree project.  An milestones shall be understandable, not too small or too large, and possible to define when it is completed or not. You can read about milestones (objectives) \href{https://coursepress.lnu.se/subject/thesis-projects/objectives/}{here}.\\
\\
\begin{tabular} {|p{0.7cm}|p{10cm}|p{1.7cm}|} \hline
\textbf{M1} & Milestone 1 ... & April XX\\ \hline
\textbf{M2} & Milestone 2 ... & May YY\\ \hline
\end{tabular}\\

\noindent
\textbf{Notice:} Consider the milestone as a concrete list of smaller problems to solve (or tasks to handle)  in order to handle the overall problems presented in section \textit{Problem formulation}.

\section*{Method}
Here you described which method you plan to use to handle your problems. Try also to motivate your choice of methods The most common ones are: Controlled Experiment, Design Science, Case Study, Systematic Literature Review, and Verification and Validation. The method is a very important part of your project proposal - think of this as the instruction manual for doing your thesis.  You can read about methods \href{https://coursepress.lnu.se/subject/thesis-projects/method-overview/}{here}.\\

\noindent
\textbf{Notice:}  Not all problems/milestones need to be handled by the same method. It is often the case that certain milestones are handled by a Literature Review whereas others are handled by a Controlled Experiment. 

\section*{Team Work}
This section is only relevant for students that work in a team.

The final grade for the thesis project is individual. That is, your Lnu supervisor, the reader, and the examinator must be able to say that each individual student has made a substantial contribution to the team effort. In order to facilitate individual grading  each part of the work should have \emph{one} responsible student. Later on, in the final report, you will be asked to tag each part of your work with the name of a team member. The team member responsible for a certain section should in some way "own" this section. That means, he/she should be able to answer detailed questions regarding this section. Also, we expect a 50-50 (roughly) division of work across the different types of project tasks. For example, both students must be involved in the technical stuff (e.g., implementation) and the more theoretical parts (e.g., writing about research methods).\\

\noindent
\textbf{Notice:} That each part of your work has one responsible team member does not mean that you are not allowed to work together on certain tasks. Feel free to work together on all parts if you think that is a good idea. However, in the end, only one of you will responsible for each part of the thesis.\\

\noindent
In this section of the project plan we want you to a make first attempt at describing how you plan to divide the work in your team. 



\newpage
\bibliographystyle{plain}
\bibliography{references}

\end{document}}
